\section{Mind Reader JAVA Program}
\subsection{Objective}
The task is to change the button name from "Submit" to "Chin Tapak Dum Dum" in the *SymbolApp.java* program into a GitHub repository.

\subsection{Program Overview}
Process to Change the Button Name:
\begin{itemize}
    \item To rename the button in the code, the label of the button needs to be changed from "Submit" to the desired new label. In the code, the following line:
    \begin{verbatim}
    submitButton = new Button("Submit");
    \end{verbatim}
    \item Can be updated to reflect the new name for the button. The updated code would be:
    \begin{verbatim}
    submitButton = new Button("Chin Tapak Dum Dum");
    \end{verbatim}
    \item The result of the updated button name will be displayed to the user.
\end{itemize}

\subsection{Code Implementation}
\begin{verbatim}
import java.awt.*;
import java.awt.event.*;
import java.util.Random;

public class SymbolApp extends Frame implements ActionListener {
    private Label[] symbolLabels = new Label[99];
    private Button submitButton;
    private String specialSymbol;
    private String selectedSymbol;

    public SymbolApp() {
        // Generate a random special symbol
        Random rand = new Random();
        specialSymbol = Character.toString((char) (rand.nextInt(94) + 33)); // Random ASCII character from 33 to 126
        selectedSymbol = specialSymbol;

        // Setting up the main frame
        setLayout(new BorderLayout());
        setSize(800, 700);
        setTitle("Symbol App");

        // Adding instruction message
        TextArea instruction = new TextArea(
            "Think of any two digit number. Now reverse it and find the difference of them.\n" +
            "Now find the number you got and remember the symbol from the panel below.\n" +
            "Don't tell me, I'll read your mind! Hit the below button when you are ready to see the magic!",
            5, 60, TextArea.SCROLLBARS_NONE);
        instruction.setEditable(false);
        instruction.setFont(new Font("Arial", Font.PLAIN, 16));
        add(instruction, BorderLayout.NORTH);

        // Panel for symbols
        Panel symbolPanel = new Panel(new GridLayout(11, 9));
        for (int i = 0; i < 99; i++) {
            String symbol = (i % 9 == 0) ? specialSymbol : Character.toString((char) (33 + (i % 94)));
            symbolLabels[i] = new Label(i + ": " + symbol); // Numbering symbols
            symbolLabels[i].setAlignment(Label.CENTER);
            symbolPanel.add(symbolLabels[i]);
        }
        add(symbolPanel, BorderLayout.CENTER);

        // Panel for submit button
        Panel controlPanel = new Panel(new FlowLayout());
        submitButton = new Button("Chin Tapak Dum Dum");
        submitButton.addActionListener(this);
        controlPanel.add(submitButton);
        add(controlPanel, BorderLayout.SOUTH);

        // Setting up the window close event
        addWindowListener(new WindowAdapter() {
            public void windowClosing(WindowEvent we) {
                System.exit(0);
            }
        });

        setVisible(true);
    }

    public void actionPerformed(ActionEvent ae) {
        // Clear the current content and display the selected symbol
        if (ae.getSource() == submitButton) {
            removeAll();
            setLayout(new BorderLayout());
            Label resultLabel = new Label(selectedSymbol, Label.CENTER);
            resultLabel.setFont(new Font("Arial", Font.BOLD, 50));
            add(resultLabel, BorderLayout.CENTER);
            validate();
            repaint();
        }
    }

    public static void main(String[] args) {
        new SymbolApp();
    }
}
\end{verbatim}

\subsection{Compiling and Running the Program}
To run the Java Program:

Prerequisites:
\begin{itemize}
    \item Ensure that the Java Development Kit (JDK) is installed. The JDK can be downloaded from Oracle's official website or installed using a package manager for the operating system in use.
    \item While not mandatory, it may be easier to work with an Integrated Development Environment (IDE) such as IntelliJ IDEA, Eclipse, or NetBeans.
\end{itemize}

Steps to Run the Program:
\begin{itemize}
    \item \textbf{Install Java}

    On Windows:
    \begin{itemize}
        \item Download the JDK from Oracle's website.
        \item Install the JDK by following the instructions.
        \item Set the \texttt{JAVA\_HOME} environment variable and update the \texttt{Path} environment variable with the JDK's bin directory.
        \item To verify the installation, open Command Prompt and run \texttt{java -version}.
    \end{itemize}

    \item \textbf{Save the Program}
    \begin{itemize}
        \item Open a text editor (e.g., Notepad, VSCode, or any Java IDE).
        \item Save the file as \texttt{SymbolApp.java}.
    \end{itemize}

    \item \textbf{Compile the Program}
    \begin{itemize}
        \item Open a terminal or command prompt.
        \item Navigate to the directory where the \texttt{SymbolApp.java} file is saved.
        \begin{verbatim}
        cd path/to/your/java/file
        \end{verbatim}
        \item Compile the program with:
        \begin{verbatim}
        javac SymbolApp.java
        \end{verbatim}
        This will generate a \texttt{SymbolApp.class} file if there are no compilation errors.
    \end{itemize}

    \item \textbf{Run the Program}
    \begin{itemize}
        \item After successful compilation, run the program using:
        \begin{verbatim}
        java SymbolApp
        \end{verbatim}
        \item The application window will open, allowing interaction as described in the program.
    \end{itemize}
\end{itemize}

\subsection{Adding the JAVA Program to GitHub Repository via GitHub Desktop}
To add this Java program to a GitHub repository using GitHub Desktop, follow these steps:
\begin{enumerate}
    \item Open \textbf{GitHub Desktop}.
    \item Clone your repository:
    \begin{enumerate}
        \item In the top-left corner, click \textbf{File} \(\rightarrow\) \textbf{Clone Repository}.
        \item Choose your repository from the list or paste the URL \texttt{https://github.com/yourusername/your-repo-name.git}.
        \item Select the local path where you want to clone the repository and click \textbf{Clone}.
    \end{enumerate}
    \item Add the \texttt{SymbolApp.java} file to your local repository folder.
    \item Return to \textbf{GitHub Desktop}.
    \item In the \textbf{Changes} tab, you should see the \texttt{SymbolApp.java} file listed.
    \item Commit the changes:
    \begin{enumerate}
        \item Enter a commit message such as \texttt{Rename button name in the JAVA program}.
        \item Click the \textbf{Commit to master} button.
    \end{enumerate}
    \item Push the changes to GitHub:
    \begin{enumerate}
        \item Click \textbf{Push origin} to upload the commit to your GitHub repository.
    \end{enumerate}
    \item Verify the upload:
    \begin{enumerate}
        \item Visit your GitHub repository URL in a web browser.
        \item Verify that the \texttt{SymbolApp.java} file is listed and accessible in the repository.
    \end{enumerate}
\end{enumerate}


\subsection{Conclusion}
\begin{itemize}
    \item The button's text has been changed from "Submit" to "Chin Tapak Dum Dum".
    \item The functionality of the button remains unchanged while updating its displayed label in the user interface.
\end{itemize}