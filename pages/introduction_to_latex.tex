\section{Introduction to Latex}
\subsection {Objective}
Latex is a high-quality typesetting system used for the creation of technical and scientific documents. It separates content from formatting, allowing users to focus on writing while Latex handles the layout. Its ability to handle complex documents, such as theses or research papers, makes it the go-to choice for many academic and professional writers.


\subsection{Why Use Latex?}
Some key benefits of using Latex{} include:
\begin{itemize}
    \item High-quality typesetting
    \item Perfect for documents containing mathematical symbols and formulas
    \item Automatic management of cross-references, citations, and tables of contents
    \item Free and cross-platform
    \item Excellent for large documents, such as books or reports
\end{itemize}

\subsection{Basic Commands in Latex}
Here are some fundamental Latex{} commands:
\begin{itemize}
    \item \texttt{\textbackslash documentclass\{article\}}: Defines the document type.
    \item \texttt{\textbackslash usepackage\{package\}}: Adds packages for extra functionality.
    \item \texttt{\textbackslash title\{Title\}}, \texttt{\textbackslash author\{Author\}}, \texttt{\textbackslash date\{\}}: Define title, author, and date.
    \item \texttt{\textbackslash section\{Section Title\}}, \texttt{\textbackslash subsection\{Subsection Title\}}: Adds sections and subsections.
    \item \texttt{\textbackslash textbf\{bold\}}, \texttt{\textbackslash textit\{italic\}}: For bold and italic text.
\end{itemize}

\subsection{Mathematical Typesetting}
Latex{} excels at handling mathematical formulas, such as the well-known Einstein equation:
\begin{equation}
    E = mc^2
\end{equation}
Inline mathematical expressions can be written like this: \( a^2 + b^2 = c^2 \).

\subsection{Steps to Compile a Latex{} Document}
Compiling a Latex{} document is the process of transforming the \texttt{.tex} source file into a formatted document, typically a PDF. Here's how you can compile your document:

\subsubsection{Writing the Document}
Write your document in a plain text editor like \texttt{Notepad}, \texttt{TeXShop} (Mac), or a dedicated LaTeX editor like \texttt{Overleaf}, \texttt{TeXworks}, or \texttt{Kile}. Save the file with a \texttt{.tex} extension.

\subsubsection{Using an Editor or Compiler}
\begin{itemize}
    \item Overleaf: Upload your \texttt{.tex} file to \href{https://www.overleaf.com}{Overleaf}, an online LaTeX editor. The document will compile automatically.
\end{itemize}

\subsubsection{Compiling the Document}
To compile the \texttt{.tex} file into a PDF:
\begin{itemize}
    \item In Overleaf: The compilation happens automatically. You can download the final PDF once it compiles.
\end{itemize}

\subsubsection{Viewing the PDF}
After compiling, your LaTeX editor or Overleaf will usually display the resulting PDF. If you're working locally, you can open the generated PDF using any PDF viewer.

\subsection{Conclusion}
This document provided an introduction to Latex and detailed steps on how to compile a document. With the right tools and a bit of practice, we will be able to create professional documents with ease.
