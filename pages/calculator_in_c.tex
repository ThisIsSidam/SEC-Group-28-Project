
\section{Calculator in C }

\subsection{Objective}
The objective of this lab is to develop a basic calculator program using the C programming language. The calculator will perform simple arithmetic operations like addition, subtraction, multiplication, and division based on user input.

\subsection{Program Overview}
The calculator program is designed to:
\begin{itemize}
    \item Accept two numbers from the user.
    \item Prompt the user to select an arithmetic operation (Addition, Subtraction, Multiplication, Division).
    \item Perform the selected operation.
    \item Display the result of the operation to the user.
\end{itemize}
The program includes error handling to manage division by zero and other invalid inputs.

\subsection{Code Implementation}
The following is the C code for the calculator program:

\begin{lstlisting}[language=C, caption={Calculator Program in C}, label={lst:calculator}]
#include <stdio.h>

int main() {
    char operator;
    double num1, num2, result;

    printf("Enter an operator (+, -, *, /): ");
    scanf(" %c", &operator);

    printf("Enter two operands: ");
    scanf("%lf %lf", &num1, &num2);

    switch(operator) {
        case '+':
            result = num1 + num2;
            break;
        case '-':
            result = num1 - num2;
            break;
        case '*':
            result = num1 * num2;
            break;
        case '/':
            if (num2 != 0)
                result = num1 / num2;
            else {
                printf("Error! Division by zero.\n");
                return -1;
            }
            break;
        default:
            printf("Error! Operator is not correct\n");
            return -1;
    }

    printf("Result: %.2lf\n", result);
    return 0;
}
\end{lstlisting}

\subsection{Initialize a Local Git Repository}
\begin{enumerate}
    \item Open \textbf{GitHub Desktop}.
    \item Click on \textbf{File} \textbf{New Repository}.
    \item Fill in the repository details:
    \begin{itemize}
        \item \textbf{}: Name the repository.
        \item \textbf{}: Choose or create a directory where the \textbf{calculator.c} file is located.
        \item \textbf{} Select a template to ignore certain files.
    \end{itemize}
    \item Click \textbf{Create Repository}. This initializes the repository and opens it in GitHub Desktop.
\end{enumerate}

\subsection{Add the File to the Repository}
\begin{enumerate}
    \item In \textbf{GitHub Desktop}, see the \texttt{calculator.c} file listed under the \textbf{Changes} tab on the left side.
    \item Add a summary and optional description for your commit in the \textbf{Summary} and \textbf{Description} fields at the bottom left.
\end{enumerate}

\subsection{Commit the Changes}
\begin{enumerate}
    \item After reviewing the changes, click \textbf{Commit to main} (or \textbf{master}, depending on the default branch name) in the lower-left corner.
\end{enumerate}

\subsection{Push the Changes to GitHub}
\begin{enumerate}
    \item After committing the changes, then publish the repository:
    \begin{itemize}
        \item Click on the \textbf{Publish repository} button in the upper-right corner.
        \item If this is a new repository, GitHub Desktop will prompt to enter a name for the remote repository and optionally a description.
        \item Click \textbf{Publish repository} to push the local commits to GitHub.
    \end{itemize}
\end{enumerate}

\subsection{Verify the Upload}
\begin{enumerate}
    \item Open the web browser and navigate to GitHub repository URL.
    \item Verify that the \texttt{calculator.c} file is listed and accessible in the repository.
\end{enumerate}

By following these steps in GitHub Desktop, one can be able to perform the same actions as in the GitHub CLI, but with a graphical user interface. 
