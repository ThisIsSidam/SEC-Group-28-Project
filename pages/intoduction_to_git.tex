
\section{What is Git?}

Git is a distributed version control system designed to manage and track changes in source code during software development. It was created by Linus Torvalds in 2005 to support the development of the Linux kernel. Git is known for its efficiency, performance, and reliability in handling projects of all sizes, from small personal projects to large-scale enterprise applications.

\section{Key Features of Git}

\begin{enumerate}
    \item \textbf{Distributed Version Control}: Every developer has a local copy of the entire project history, including all branches and commits. This allows for fast access to project data and enables developers to work offline.
    \item \textbf{Branching and Merging}: Git allows developers to create branches to work on different features or bug fixes independently. Branches can be merged back into the main codebase once the work is complete, enabling a smooth integration process.
    \item \textbf{Commit History}: Changes in Git are recorded as commits. Each commit represents a snapshot of the project at a specific point in time and includes a unique identifier and a message describing the changes. This history is crucial for tracking the evolution of the project and understanding the rationale behind changes.
    \item \textbf{Staging Area}: Git uses a staging area to prepare changes for a commit. Developers can add specific changes to the staging area, allowing them to group related changes into a single commit.
    \item \textbf{Collaboration}: Git supports collaborative workflows through features like pull requests and code reviews. Pull requests allow contributors to propose changes, which can be reviewed and discussed before being merged into the main codebase.
    \item \textbf{Conflict Resolution}: Git provides tools to help manage and resolve conflicts that arise when multiple developers make changes to the same part of the codebase.
    \item \textbf{Performance}: Git is optimized for performance, handling large projects and repositories efficiently. It uses compression techniques to minimize the storage required for the project history.
\end{enumerate}

\section{Using Git and GitHub with GitHub Desktop}

\subsection{Objective}

To understand how to use Git and GitHub through the GitHub Desktop application for version control and collaborative development.

\subsection{Materials Required}

\begin{itemize}
    \item A computer with internet access
    \item GitHub Desktop installed (\href{https://desktop.github.com/}{Download GitHub Desktop})
    \item A GitHub account (\href{https://github.com/join}{Create a GitHub account})
\end{itemize}

\subsection{Experiment Steps}

\subsubsection{Installing GitHub Desktop}

\begin{enumerate}
    \item \textbf{Download GitHub Desktop}:
    \begin{itemize}
        \item Go to the \href{https://desktop.github.com/}{GitHub Desktop website}.
        \item Download the installer for your operating system (Windows or macOS).
    \end{itemize}
    
    \item \textbf{Install the Application}:
    \begin{itemize}
        \item Run the downloaded installer and follow the on-screen instructions to install GitHub Desktop.
    \end{itemize}
\end{enumerate}

\subsubsection{Setting Up GitHub Desktop}

\begin{enumerate}
    \item \textbf{Open GitHub Desktop}:
    \begin{itemize}
        \item Launch GitHub Desktop after installation.
    \end{itemize}

    \item \textbf{Sign In}:
    \begin{itemize}
        \item Sign in with your GitHub account credentials. If you don’t have an account, create one at \href{https://github.com/join}{GitHub}.
    \end{itemize}
\end{enumerate}

\subsubsection{Cloning a Repository}

\begin{enumerate}
    \item \textbf{Clone a Repository}:
    \begin{itemize}
        \item Click on "File" in the menu bar.
        \item Select "Clone Repository."
    \end{itemize}

    \item \textbf{Choose Repository Source}:
    \begin{itemize}
        \item Select a repository from GitHub.com, a GitHub Enterprise instance, or enter a repository URL.
    \end{itemize}

    \item \textbf{Select Local Path}:
    \begin{itemize}
        \item Choose the local path where you want to clone the repository.
    \end{itemize}

    \item \textbf{Clone}:
    \begin{itemize}
        \item Click "Clone" to download the repository to your local machine.
    \end{itemize}
\end{enumerate}

\subsubsection{Creating a New Repository}

\begin{enumerate}
    \item \textbf{Create Repository}:
    \begin{itemize}
        \item Click on "File" in the menu bar.
        \item Select "New Repository."
    \end{itemize}

    \item \textbf{Fill Repository Details}:
    \begin{itemize}
        \item Enter the repository name and description.
        \item Choose the local path for the repository.
        \item Optionally, initialize with a README file, .gitignore, and a license.
    \end{itemize}

    \item \textbf{Create}:
    \begin{itemize}
        \item Click "Create Repository."
    \end{itemize}
\end{enumerate}

\subsubsection{Making Changes and Committing}

\begin{enumerate}
    \item \textbf{Edit Files}:
    \begin{itemize}
        \item Open the repository in your preferred text editor or IDE and make changes.
    \end{itemize}

    \item \textbf{View Changes in GitHub Desktop}:
    \begin{itemize}
        \item Return to GitHub Desktop to see the list of modified files under the "Changes" tab.
    \end{itemize}

    \item \textbf{Stage Changes}:
    \begin{itemize}
        \item Select the files you want to commit.
    \end{itemize}

    \item \textbf{Write Commit Message}:
    \begin{itemize}
        \item Enter a commit message describing your changes.
    \end{itemize}

    \item \textbf{Commit}:
    \begin{itemize}
        \item Click "Commit to main" (or the current branch name).
    \end{itemize}
\end{enumerate}

\subsubsection{Pushing Changes to GitHub}

\begin{enumerate}
    \item \textbf{Push Changes}:
    \begin{itemize}
        \item After committing your changes, click "Push origin" to upload your changes to the remote repository on GitHub.
    \end{itemize}
\end{enumerate}

\subsubsection{Creating and Switching Branches}

\begin{enumerate}
    \item \textbf{Create a New Branch}:
    \begin{itemize}
        \item Click on the current branch name in the top bar.
        \item Select "New Branch."
        \item Enter a name for the new branch and click "Create Branch."
    \end{itemize}

    \item \textbf{Switch Branches}:
    \begin{itemize}
        \item To switch branches, click on the branch name and select the desired branch.
    \end{itemize}
\end{enumerate}

\subsubsection{Creating a Pull Request}

\begin{enumerate}
    \item \textbf{Push Branch to GitHub}:
    \begin{itemize}
        \item Push the branch to GitHub after making changes.
    \end{itemize}

    \item \textbf{Create Pull Request}:
    \begin{itemize}
        \item Click "Create Pull Request" in GitHub Desktop.
        \item This opens a web page to add a title, description, and reviewers for your pull request.
    \end{itemize}

    \item \textbf{Submit Pull Request}:
    \begin{itemize}
        \item Click "Create Pull Request" on the website to submit it.
    \end{itemize}
\end{enumerate}

\subsubsection{Merging Pull Requests}

\begin{enumerate}
    \item \textbf{Review Pull Request}:
    \begin{itemize}
        \item Go to the pull request on GitHub’s website.
    \end{itemize}

    \item \textbf{Merge}:
    \begin{itemize}
        \item Click "Merge Pull Request" and confirm the merge.
    \end{itemize}
\end{enumerate}

\section{Observations and Results}

\begin{itemize}
    \item Successfully installed and set up GitHub Desktop.
    \item Cloned existing repositories and created new repositories.
    \item Made and committed changes to local repositories.
    \item Pushed changes to remote repositories on GitHub.
    \item Created and managed branches.
    \item Created and merged pull requests.
\end{itemize}

\section{Conclusion}

Using Git and GitHub through GitHub Desktop simplifies version control and collaborative development by providing a user-friendly interface for managing repositories, making commits, and handling pull requests.

