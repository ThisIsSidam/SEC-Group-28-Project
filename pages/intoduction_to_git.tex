\section{Introduction To Git}

\subsection{Objective}
Git is a distributed version control system designed to manage and track changes in source code during software development. It was created by Linus Torvalds in 2005 to support the development of the Linux kernel. Git is known for its efficiency, performance, and reliability in handling projects of all sizes, from small personal projects to large-scale enterprise applications.

\subsection{Key Features of Git}

\begin{enumerate}
    \item \textbf{Distributed Version Control}: Every developer has a local copy of the entire project history, including all branches and commits. This allows for fast access to project data and enables developers to work offline.
    \item \textbf{Branching and Merging}: Git allows developers to create branches to work on different features or bug fixes independently. Branches can be merged back into the main codebase once the work is complete, enabling a smooth integration process.
    \item \textbf{Commit History}: Changes in Git are recorded as commits. Each commit represents a snapshot of the project at a specific point in time and includes a unique identifier and a message describing the changes. This history is crucial for tracking the evolution of the project and understanding the rationale behind changes.
    \item \textbf{Staging Area}: Git uses a staging area to prepare changes for a commit. Developers can add specific changes to the staging area, allowing them to group related changes into a single commit.
    \item \textbf{Collaboration}: Git supports collaborative workflows through features like pull requests and code reviews. Pull requests allow contributors to propose changes, which can be reviewed and discussed before being merged into the main codebase.
    \item \textbf{Conflict Resolution}: Git provides tools to help manage and resolve conflicts that arise when multiple developers make changes to the same part of the codebase.
    \item \textbf{Performance}: Git is optimized for performance, handling large projects and repositories efficiently. It uses compression techniques to minimize the storage required for the project history.
\end{enumerate}

\subsection{Using Git and GitHub with GitHub Desktop}

\subsubsection{Materials Required}

\begin{itemize}
    \item A computer with internet access
    \item GitHub Desktop installed (\href{https://desktop.github.com/}{Download GitHub Desktop})
    \item A GitHub account (\href{https://github.com/join}{Create a GitHub account})
\end{itemize}

\subsubsection{Experiment Steps}

\paragraph{Installing GitHub Desktop}
\begin{enumerate}
    \item Download and install GitHub Desktop from \href{https://desktop.github.com/}{here}.
\end{enumerate}

\paragraph{Setting Up GitHub Desktop}
\begin{enumerate}
    \item Open GitHub Desktop and sign in with your GitHub account.
\end{enumerate}

\paragraph{Cloning a Repository}
\begin{enumerate}
    \item Select "Clone Repository" from the "File" menu.
    \item Choose a repository from GitHub or enter the repository URL.
\end{enumerate}

\paragraph{Creating a New Repository}
\begin{enumerate}
    \item Select "New Repository" from the "File" menu.
    \item Enter repository details and click "Create."
\end{enumerate}

\paragraph{Making Changes and Committing}
\begin{enumerate}
    \item Edit files in your preferred editor.
    \item Stage and commit changes in GitHub Desktop.
\end{enumerate}

\paragraph{Pushing Changes to GitHub}
\begin{enumerate}
    \item Click "Push origin" to upload changes to GitHub.
\end{enumerate}

\paragraph{Creating and Switching Branches}
\begin{enumerate}
    \item Click the branch name to create or switch branches.
\end{enumerate}

\paragraph{Creating a Pull Request}
\begin{enumerate}
    \item Push the branch, then click "Create Pull Request" in GitHub Desktop.
\end{enumerate}

\paragraph{Merging Pull Requests}
\begin{enumerate}
    \item Review and merge the pull request on GitHub’s website.
\end{enumerate}

\subsection{Observations and Results}

\begin{itemize}
    \item Successfully installed and set up GitHub Desktop.
    \item Cloned existing repositories and created new repositories.
    \item Made and committed changes to local repositories.
    \item Pushed changes to remote repositories on GitHub.
    \item Created and managed branches.
    \item Created and merged pull requests.
\end{itemize}

\subsection{Conclusion}

Using Git and GitHub through GitHub Desktop simplifies version control and collaborative development by providing a user-friendly interface for managing repositories, making commits, and handling pull requests.
